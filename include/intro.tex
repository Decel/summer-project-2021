\section{Введение}
Математическая нейронаука это активно развивающаяся область, занимающаяся разработкой и эксплуатацией математических и вычислительных подходов для решения вопросов сетевой нейробиологии. Она изучает нейронное кодирование и нейронные сети используя новейшие методы алгебры, топологии и геометрии. Мотивацией нашей работы является исследование Владимира Ицкова, формирующее гипотезу об гомотопической эквивалентности конструкции, строящей по облаку точек симплициальное частично упорядоченное множество, букету сфер. Мы хотели посредством вычислительного эксперимента проверить гипотезу в частном случае.