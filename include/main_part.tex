\section{Основная часть}

\subsection{Основные определения}
\begin{definition}
 {\it Симплициальным комплексом} на конечном множестве вершин $M$ называется совокупность $K \subset 2^{M}$ подмножеств множества $M$, удовлетворяющая следующим двум условиям:
    \begin{enumerate}
        \item если $I \in K$ и $J \subset I$, то $J \in K$;
        \item $\varnothing \in K$.
    \end{enumerate}
\end{definition}

\begin{definition}
{\it Симплексом} называются элементы симплециального комплекса $K$.
\end{definition}

\begin{definition}
$V$ это конечное множество вершин
\begin{itemize}
    \item Последовательность в $V$ это симплекс с линейным порядком 
    \item Множество всех последовательностей это частично упорядоченное множество (чум далее)
\end{itemize}
\end{definition}

\begin{definition}
{\it Направленный Комплекс} это чум последовательностей в $V$, закрытых на включении.
\end{definition}

\begin{definition}
{\it Гипотеза.} Пусть $x \subset \mathbb{R}^d$ будет множеством точек в общем положении. Положим, что либо $d \leq 3$ и $n \geq d + 2$ или $d \geq 4$ и $n \geq 2d - 1$. Тогда гомология направленного комплекса $D_{lin}(X)$ удовлетворяет $H_{*}(D_{lin}(X)) = H_{*}(\bigvee^{n - 1} S^d)$ 
\end{definition}

\subsection{Реализация}

Для начала, мы решили попробовать написать собственную программу для вычисления гомологий симплициальных комплексов, используя нормальную форму Смита, язык Python и пакет sympy. После чего приступили к изучению работы популярных пакетов по работе с симплициальными множествами, таких как Simplicial из репозитория Nebneuron лаборатории института математической нейробиологии, работающий с такими реализациями как  Persistent Homology Algorithm Toolbox (PHAT) и The Perseus Software Project for Rapid Computation of Persistent Homology. (Perseus). Все реализации являются крайне интересными, однако только Simplicial использует язык Julia наиболее подходящий для объемных вычислений, в свою очередь я бы хотел продолжить работать С Антоном Андреевичем над данным проектом и попробовать реализовать аналогичный пакет на Scala. 



\subsection{Описание полученных результатов}
Результаты практики:
\begin{itemize}
    \item Были изучены материалы связанные с вычислительной топологией
    \item Мы ознакомились с работой Владимира Ицкова
    \item Была реализована программа считывающая гомологии симплициального комплекса посредством нормальной формы Смита
    \item Мы изучили пакеты реализующие вычисления гомологий симплициальных чумов
    \item Мы попрактиковались в вычислительной топологии и поверхностно ознакомились с математической нейробиологией
\end{itemize}