% Это основная команда, с которой начинается любой \LaTeX-файл. Она отвечает за тип документа, с которым связаны основные правил оформления текста.
\documentclass[a4paper, 17pt]{article}

% Здесь идет преамбула документа, тут пишутся команды, которые настраивают LaTeX окружение, подключаете внешние пакеты, определяете свои команды и окружения. В данном случае я это делаю в отдельных файлах, а тут подключаю эти файлы.

% Здесь я подключаю разные стилевые пакеты. Например возможности набирать особые символы или возможность компилировать русский текст. Подробное описание внутри.
\usepackage{packages}

% Здесь я определяю разные окружения, например, теоремы, определения, замечания и так далее. У этих окружений разные стили оформления, кроме того, эти окружения могут быть нумерованными или нет. Все подробно объяснено внутри.
\usepackage{environments}

% Здесь я определяю разные команды, которых нет в LaTeX, но мне нужны, например, команда \tr для обозначения следа матрицы. Или я переопределяю LaTeX команды, которые работают не так, как мне хотелось бы. Типичный пример мнимая и вещественная часть комплексного числа \Im, \Re. В оригинале они выглядят не так, как мы привыкли. Кроме того, \Im еще используется и для обозначения образа линейного отображения. Подробнее описано внутри.
\usepackage{commands}

% Пакет для титульника проекта
\usepackage{titlepage}

% Здесь задаем параметры титульной страницы
\setUDK{192.168.1.1}
% Выбрать одно из двух
\setToResearch
%\setToProgram

\setTitle{НИУ ВШЭ}

% Выбрать одно из трех:
% КТ1 -- \setStageOne
% КТ2 -- \setStageTwo
% Финальная версия -- \setStageFinal
%\setStageOne
%\setStageTwo
%\setStageFinal

\setGroup{198}
%сюда можно воткнуть картинку подписи
\setStudentSgn{}
\setStudent{Мкртчян О. С.}
\setStudentDate{5.09.2021}
\setAdvisor{Айзенберг Антон Андреевич}
\setAdvisorTitle{МЛ алгебраической топологии и ее приложений, завлаб}
\setAdvisorAffiliation{ФКН НИУ ВШЭ}
\setAdvisorDate{5.09.2021}
\setGrade{}
%сюда можно воткнуть картинку подписи
\setAdvisorSgn{}
\setYear{2021}


% С этого момента начинается текст документа
\begin{document}

    \setcounter{page}{2}
    
    \tableofcontents

    % Это основная команда, с которой начинается любой \LaTeX-файл. Она отвечает за тип документа, с которым связаны основные правил оформления текста.
\documentclass[a4paper, 17pt]{article}

% Здесь идет преамбула документа, тут пишутся команды, которые настраивают LaTeX окружение, подключаете внешние пакеты, определяете свои команды и окружения. В данном случае я это делаю в отдельных файлах, а тут подключаю эти файлы.

% Здесь я подключаю разные стилевые пакеты. Например возможности набирать особые символы или возможность компилировать русский текст. Подробное описание внутри.
\usepackage{packages}

% Здесь я определяю разные окружения, например, теоремы, определения, замечания и так далее. У этих окружений разные стили оформления, кроме того, эти окружения могут быть нумерованными или нет. Все подробно объяснено внутри.
\usepackage{environments}

% Здесь я определяю разные команды, которых нет в LaTeX, но мне нужны, например, команда \tr для обозначения следа матрицы. Или я переопределяю LaTeX команды, которые работают не так, как мне хотелось бы. Типичный пример мнимая и вещественная часть комплексного числа \Im, \Re. В оригинале они выглядят не так, как мы привыкли. Кроме того, \Im еще используется и для обозначения образа линейного отображения. Подробнее описано внутри.
\usepackage{commands}

% Пакет для титульника проекта
\usepackage{titlepage}

% Здесь задаем параметры титульной страницы
\setUDK{192.168.1.1}
% Выбрать одно из двух
\setToResearch
%\setToProgram

\setTitle{НИУ ВШЭ}

% Выбрать одно из трех:
% КТ1 -- \setStageOne
% КТ2 -- \setStageTwo
% Финальная версия -- \setStageFinal
%\setStageOne
%\setStageTwo
%\setStageFinal

\setGroup{198}
%сюда можно воткнуть картинку подписи
\setStudentSgn{}
\setStudent{Мкртчян О. С.}
\setStudentDate{5.09.2021}
\setAdvisor{Айзенберг Антон Андреевич}
\setAdvisorTitle{МЛ алгебраической топологии и ее приложений, завлаб}
\setAdvisorAffiliation{ФКН НИУ ВШЭ}
\setAdvisorDate{5.09.2021}
\setGrade{}
%сюда можно воткнуть картинку подписи
\setAdvisorSgn{}
\setYear{2021}


\begin{document}

\\\\
Задачи практики:
\begin{itemize}
    \item Разобрать конструкцию Ицкова
    \item Подготовить код, вычисляющий гомологии симплициального чума
    \item Повычислять гомологии чумов Ицкова
    
    
\end{itemize}



\end{document}
    \section{Календарный план-график}


\begin{table}[h!]
\centering
\begin{tabular}{ | m{0.75cm} | m{3.5cm} | m{10.5cm} |}
\hline

№ п/п & Сроки проведения & Выполненные работы   \\
\hline
1 & $1.07.21$ &  Инструктаж по ознакомлению с требованиями охраны труда, техники безопасности, пожарной безопасности, а так же правилами внутреннего трудового распорядка  \\ 
\hline 
2 & $02.07-07.07.21$ &  Повторное ознакомление с методичкой по гомологиям   \\
\hline
3 & $03.07.2021$ & Изучение популярных реализаций вычислений гомологий симплициальных комплексов.  \\
\hline
4 & $04-10.07.2021$ &  Подготовка кода для вычисления гомологий.  \\
\hline
5 & $10-14.07.2021$ & Попытки разобраться с чумами Ицкова и их реализацией. \\
\hline
6 & $15.07.2021$ & Подготовка отчета по практике \\
\hline
\end{tabular}
\end{table}
    \section{Введение}
Математическая нейронаука это активно развивающаяся область, занимающаяся разработкой и эксплуатацией математических и вычислительных подходов для решения вопросов сетевой нейробиологии. Она изучает нейронное кодирование и нейронные сети используя новейшие методы алгебры, топологии и геометрии. Мотивацией нашей работы является исследование Владимира Ицкова, формирующее гипотезу об гомотопической эквивалентности конструкции, строящей по облаку точек симплициальное частично упорядоченное множество, букету сфер. Мы хотели посредством вычислительного эксперимента проверить гипотезу в частном случае.
    \section{Основная часть}

\subsection{Основные определения}
\begin{definition}
 {\it Симплициальным комплексом} на конечном множестве вершин $M$ называется совокупность $K \subset 2^{M}$ подмножеств множества $M$, удовлетворяющая следующим двум условиям:
    \begin{enumerate}
        \item если $I \in K$ и $J \subset I$, то $J \in K$;
        \item $\varnothing \in K$.
    \end{enumerate}
\end{definition}

\begin{definition}
{\it Симплексом} называются элементы симплециального комплекса $K$.
\end{definition}

\begin{definition}
$V$ это конечное множество вершин
\begin{itemize}
    \item Последовательность в $V$ это симплекс с линейным порядком 
    \item Множество всех последовательностей это частично упорядоченное множество (чум далее)
\end{itemize}
\end{definition}

\begin{definition}
{\it Направленный Комплекс} это чум последовательностей в $V$, закрытых на включении.
\end{definition}

\begin{definition}
{\it Гипотеза.} Пусть $x \subset \mathbb{R}^d$ будет множеством точек в общем положении. Положим, что либо $d \leq 3$ и $n \geq d + 2$ или $d \geq 4$ и $n \geq 2d - 1$. Тогда гомология направленного комплекса $D_{lin}(X)$ удовлетворяет $H_{*}(D_{lin}(X)) = H_{*}(\bigvee^{n - 1} S^d)$ 
\end{definition}

\subsection{Реализация}

Для начала, мы решили попробовать написать собственную программу для вычисления гомологий симплициальных комплексов, используя нормальную форму Смита, язык Python и пакет sympy. После чего приступили к изучению работы популярных пакетов по работе с симплициальными множествами, таких как Simplicial из репозитория Nebneuron лаборатории института математической нейробиологии, работающий с такими реализациями как  Persistent Homology Algorithm Toolbox (PHAT) и The Perseus Software Project for Rapid Computation of Persistent Homology. (Perseus). Все реализации являются крайне интересными, однако только Simplicial использует язык Julia наиболее подходящий для объемных вычислений, в свою очередь я бы хотел продолжить работать С Антоном Андреевичем над данным проектом и попробовать реализовать аналогичный пакет на Scala. 



\subsection{Описание полученных результатов}
Результаты практики:
\begin{itemize}
    \item Были изучены материалы связанные с вычислительной топологией
    \item Мы ознакомились с работой Владимира Ицкова
    \item Была реализована программа считывающая гомологии симплициального комплекса посредством нормальной формы Смита
    \item Мы изучили пакеты реализующие вычисления гомологий симплициальных чумов
    \item Мы попрактиковались в вычислительной топологии и поверхностно ознакомились с математической нейробиологией
\end{itemize}
    \section{Заключение}
В рамках выполнения данной работы студент получил знания по вычислительной топологии и симлпициальным гомологиям, изучил современные пакеты работы с симплициальными множествами, реализовал программу для расчёта гомологий симплициального комплекса. 

    \section{Список источников}

\begin{enumerate}
    \item  \href{https://github.com/AntonAyzenberg/Persistent-homology-Examples/blob/master/My_Persistence_test.ipynb}{А. А. Айзенберг, пример вычисления устойчивых гомологий в Python}.
    \item {А. А. Айзенберг, Комбинаторика, топология и алгебра симплициальных комплексов, конспект курса НОЦ Миан}.
    \item \href{https://vk.com/doc3973145_562177830?hash=a367bc57201051240d&dl=f3e20e16bcf57bdb73}{А. А. Айзенберг, Методичка по симплициальным комплексам и гомологиям}.
    \item \href{https://folk.ntnu.no/mariusth/Abel/Itskov.pdf} {Vladimir Itskov: Directed complexes, sequence dimension and inverting a neural network}
    \item \href{https://eric-bunch.github.io/blog/calculating_homology_of_simplicial_complex} {Calculating Homology of a Simplicial Complex Using Smith Normal Form}
    \item \href{https://github.com/nebneuron/Simplicial.jl} {Package Simplicial}
    \item {H.Edelsbrunner, J.Harer: Computational Topology, An Introduction. American Mathematical Society, 2010, ISBN 0-8218-4925-5}
\end{enumerate}
    \printbibliography

\end{document}